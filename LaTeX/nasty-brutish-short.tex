\def\year{2017}\relax
%File: formatting-instruction.tex
\documentclass[letterpaper]{article}
\usepackage{aaai17}
\usepackage{times}
\usepackage{helvet}
\usepackage{courier}
\frenchspacing
\setlength{\pdfpagewidth}{8.5in}
\setlength{\pdfpageheight}{11in}
\pdfinfo{
/Title (Nasty, Brutish, and Short: 
What Makes Election News Popular on Twitter?)
/Author (Anonymous)}
\setcounter{secnumdepth}{0}  
 \begin{document}
% The file aaai.sty is the style file for AAAI Press 
% proceedings, working notes, and technical reports.
%
\title{Nasty, Brutish, and Short: 
What Makes Election News Popular on Twitter?}
\author{Anonymous\\
%Association for the Advancement of Artificial Intelligence\\
%2275 East Bayshore Road, Suite 160\\
%Palo Alto, California 94303\\
}
\maketitle
\begin{abstract}
AAAI creates proceedings, working notes, and technical reports directly from electronic source furnished by the authors. To ensure that all papers in the publication have a uniform appearance, authors must adhere to the following instructions. 
\end{abstract}

 
\section{Introduction}
In the changing landscape of both journalism and politics, social media is playing an increasingly large role in mobilizing and spreading information to citizens. A Pew Research survey from August 2015 showed that nearly two-thirds of adults in the U.S. who are on Twitter use the platform to get news \cite{pew-Twitter-news}. During the 2016 election year, the New York Times published an article estimating a 2 billion-dollar advantage in free media, including social media, for Donald Trump, all of which has no small impact on the messages broadcast to voters [13].

Although social media messages are less able to be carefully controlled in comparison to paid advertisements, they also have the potential to reach a wider audience. Reactions to an article shared by one potential voter now have the ability to be broadcast and spread to millions of others in a real-time, public sphere. 

%Measuring the way that stories grow and spread has become an important part of the equation to understanding public discourse.

The popularity of sharing articles on social media also marks an important shift in the role of the news consumer from armchair reader to information propagator. Whereas news used to be broadcast to the reader, now each reader has the potential to broadcast stories to his or her own audience. Sharing a story requires a level of interest and activation on the part of the reader beyond simply reading a story; yet often, this trigger is predictably emotional in nature. 

In a 2011 study of the New York Time’s ``most emailed list'', Berger and Milkman found that the potential for a news story to go viral is partially driven by physiological arousal, defined as ``an excitatory state of sensory alertness, mobilization, or energy'' [Milkman 7]. In short, the desire to share a certain story is often universally impulsive, regardless of context. Yet in the case of political news, this impulse can have a large impact on reach of political messages.
 

 \subsection{Hypotheses}
With the impact of social media in politics  \emph{and} the often emotionally charged nature of content-sharing in mind, we ask the following question in our research: 

\begin{itemize}
\item Does the emotional vocabulary of political news stories have an impact on its Twitter popularity that persists beyond political affiliation?  
\end{itemize}

To test this question, we focus on three key aspects of stories: length, emotionality, and positivity, based on behavioral theories of the Internet and studies of political news detailed in the section below.  

We hypothesize the following behavior in our dataset of stories and tweets:

\begin{itemize} 
    \item \textbf{H1:} Story length has a \emph{negative} correlation with Twitter shares, due to the effects of the Internet attention economy and overexposure to political media \cite{goldhaber1997attention}.
    \item \textbf{H2:} Emotionality has a \emph{positive} correlation with Twitter shares, consistent for viral content in general \cite{berger2012makes}.
    \item \textbf{H3:} Positivity has a \emph{negative} correlation with Twitter shares, due to the nature of political news and contrary to generalized findings \cite{berger2012makes}

\end{itemize}

For each of these three independent variables (story length, emotionality, positivity) we repeat analyses across three views of the data: first, the entire dataset; then, by political candidate followed amongst users who follow only one candidate; and finally, by the number of political candidates followed (degree of political engagement), to look for differences amongst different populations of political tweeters.


\section{Literature Review}
\subsection{The Social Media Megaphone}
In the changing landscape of both journalism and politics, social media is playing an increasingly large role in mobilizing and spreading information to citizens. President Barack Obama’s win in 2008 is often attributed as the first example of a successful social media campaign in the elections. Establishing an online presence that recruited more than 3 million individual contributors and 5 million volunteers, Obama created a grassroots political movement \cite{cogburn2011networked}. Publicity and public sound bites matter-- especially when it’s free and has the potential to go viral.

This election cycle, in particular, already shows a heavy skew by social media. The New York Times estimated a 2 billion-dollar advantage in free media for Donald Trump on platforms from television to Twitter, all of which has no small impact on the messages broadcast to voters \cite{nyt-trump-free-media}. Although ``free media'' messages have less ability to be carefully controlled in comparison to paid advertisements, they also have more potential to reach a wider audience. Sentiments echoed by one potential voter now have the ability to be broadcast and spread to millions of others in a real-time, public sphere.
  
\subsection{The (Short) Attention Economy}
At the same time that social media has the power to create a flood of free advertising and media for political candidates, the abundance of information on the web has created new challenges and questions about the kind of content being processed by readers. This paradox-- between the ease of accessibility to information and the increasingly limited bandwidth of consumers-- is described as one of the challenges of being in an \emph{attention economy} \cite{goldhaber1997attention}. Moreover, high-impact events like the presidential elections especially intensifies this effect-- about 60 \% of Americans reported feeling exhausted by media coverage of the elections in July of 2016 \cite{election-fatigue}. To explore the effects of the attention economy on the reading of political news, we examine \textbf{story length} and how it relates to sharing popularity in the analysis to follow.


\subsection{Negativity in Politics and the Internet}
In addition, the option of anonymity and pseudo-anonymity on a social network like Twitter (along with other traits of Internet communication), is theorized to contribute to increased negative and hostile behavior, potentially increasing tension for the already-fraught subject of politics. This phenomenon, is coined as the \emph{online disinhibition effect} \cite{suler2004online}. 

In Berger and Milkman’s study of story virality, it was found that \emph{positive} content was more likely to be shared than negative content-- against conventional belief \cite{berger2012makes}. Political news, however, is a unique category of news, and this election in particular-- where one-in-four Americans report disliking the presidential candidates-- appears to have a negative overtone.

To compare the sharing of election news stories versus patterns of general virality in the news, and to examine the extent in which negative sentiment is popular, we calculate the \emph{negativity} of stories, and how that relates to Twitter behavior.

We also examine the effects of the degree of combined emotionality in the content and how that relates to Twitter shares, to see if either more positive or more negative content is more likely to be shared overall than content that ranks low in emotionality. Although positive content was found to be more popular than negative content in the sharing of stories, both highly positive and highly negative content was more likely to become viral, and we expect the same to hold for political news \cite{berger2012makes}. 







\bibliographystyle{aaai} \bibliography{nasty-brutish-short.bib}

\end{document}
